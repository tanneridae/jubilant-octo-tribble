\documentclass[handout]{ximera}

\title{This is the sample!}
\author{A Ximera Author}

\begin{abstract}
  This is a place to get started.
\end{abstract}

\begin{document}
\maketitle

Here's a sample question.  I am editing this document.  I am really editing.

\begin{tabular}{ | l | c | r | }
  1 & 2 & 3 \\
  4 & 5 & 6 \\
  7 & 8 & 9 \\
\end{tabular}

\begin{problem} Where is the velociraptor?
\begin{multipleChoice}
\choice[correct]{\includegraphics[scale=.2]{fluffy.png}}
\choice{Not this one!}
\choice{Click here?}
\choice{Not me!}
\end{multipleChoice}
\end{problem}

\begin{problem}
\begin{selectAll}
\choice[correct]{\includegraphics[scale=.2]{fluffy.png}}
\choice{Not this one!}
\choice[correct]{Click here?}
\choice{Not me!}
\end{selectAll}
\end{problem}

\begin{problem}
   You can test that $x + x = \answer{2x}$ or that $x \cdot x = \answer{x^2}$.
\end{problem}

\begin{problem}
   The tolerance 0.01 means $\pi \approx \answer[tolerance=0.01]{3.141592653}$
\end{problem}

\begin{problem}
   The tolerance 17 means $3421 \approx \answer[tolerance=17]{3421}$
\end{problem}

\begin{tikzpicture}
\draw (0,0) -- (1,1);
\end{tikzpicture}
\end{document}
